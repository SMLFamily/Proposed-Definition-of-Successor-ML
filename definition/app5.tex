%!TEX root = root.tex
%
\section{Appendix: The Development of ML}
\label{story-app}

This Appendix records the main stages in the development of ML, and the people
principally involved.  The main emphasis is upon the design of the language;
there is also a section devoted to implementation. On the other hand, no
attempt is made to record work on applications of the language.
\note{\thenewpreface}{This appendix has been revised and extended in
several ways. A detailed list of changes is not available.}
\subsection*{Origins}

ML and its semantic description have evolved over a period of about
twenty years.  It is a fusion of many ideas from many people;  in this
appendix we try to record and to acknowledge the important precursors
of its ideas, the important influences upon it, and the important
contributions to its design, implementation and semantic description.

ML, which stands for {\em meta language}, was conceived as a medium for
finding and performing proofs in a formal logical system.  This application
was the focus of the initial design effort, by Robin Milner in collaboration
first with Malcolm Newey and Lockwood Morris, then with Michael Gordon and
Christopher Wadsworth~\cite{GMMNW}. The intended application to proof affected
the design considerably.  Higher order functions in full generality seemed
necessary for programming proof tactics and strategies, and also a robust type
system (see below).  At the same time, imperative features were important for
practical reasons; no-one had experience of large useful programs written in a
pure functional style. In particular, an exception-raising mechanism was
highly desirable for the natural presentation of tactics.

The full definition of this first version of ML was included in a
book~\cite{GMW} which describes LCF, the proof system which ML was designed to
support.  The details of how the proof application exerted an influence on
design is reported by Milner~\cite{Mil2}.  Other early influences were the
applicative languages already in use in Artificial Intelligence, principally
LISP~\cite{McC}, ISWIM~\cite{Lan} and POP2~\cite{BP}.

\subsection*{Polymorphic types}

The polymorphic type discipline and the associated type-assignment algorithm
were\linebreak
 prompted by the need for security;  it is vital to know that when
a program produces an object which it claims to be a theorem, then it
is indeed a theorem.  A type discipline provides the security, but a
polymorphic discipline also permits considerable flexibility.

The key ideas of the type discipline were evolved in combinatory logic by
Haskell Curry and Roger Hindley, who arrived at different but equivalent
algorithms for computing principal type schemes.  Curry's~\cite{Cur} algorithm
was by equation-solving; Hindley~\cite{Hin} used the unification algorithm of
Alan Robinson~\cite{Rob} and also presented the precursor of our type
inference system.  James Morris~\cite{Mor} independently gave an
equation-solving algorithm very similar to Curry's.  The idea of an algorithm
for finding principal type schemes is very natural and may well have been
known earlier. Roger Hindley has pointed out that Carew
Meredith's inference rule for propositional logic called Condensed Detachment,
defined in the early 1950s, clearly suggests that he knew such an algorithm
\cite{Mer}.

Milner~\cite{Mil1}, during the design of ML, rediscovered principal types and
their calculation by unification, for a language (slightly richer than
combinatory logic) containing local declarations.  He and Damas~\cite{DM}
presented the ML type inference systems following Hindley's style.  Damas
\cite{Dam}, using ideas from Michael Gordon, also devised the first
mathematical treatment of polymorphism in the presence of references and
assignment.  Tofte~\cite{Tof-a} produced a different scheme employing
so-called {\em imperative types}, which was adopted in the original version of
the language.  This approach has been superseded in the present language by a
simpler scheme, suggested by Tofte~\cite{Tof-a}, Andrew Wright~\cite{wright95}, and
Xavier Leroy~\cite{Ler1}, according to which polymorphic bindings are
restricted to non-expansive expressions.

\subsection*{Refinement of the Core Language}

Two movements led to the re-design of ML.  One was the work of Rod
Burstall and his group on specifications, crystallised in the specification
language CLEAR~\cite{BG} and in the functional programming language HOPE
\cite{BMS};  the latter was for expressing executable specifications.  The
outcome of this work which is relevant here was twofold.  First, there were
elegant programming features in HOPE, particularly pattern matching and
clausal function definitions; second, there were ideas on modular construction
of specifications, using signatures in the interfaces.  A smaller but
significant movement was by Luca Cardelli, who extended the data-type
repertoire in ML by adding named records and variant types.

In 1983, Milner (prompted by Bernard Sufrin) wrote the first draft of a
standard form of ML attempting to unite these ideas; over the next three years
it evolved into the Standard ML core language.  Notable here was the harmony
found among polymorphism, HOPE patterns and Cardelli records, and the nice
generalisations of ML exceptions due to ideas from Alan Mycroft, Brian Monahan
and Don Sannella.  A simple stream-based I/O mechanism was developed from
ideas of Cardelli by Milner and Harper.  The Standard ML core language is
described in detail in a composite report~\cite{HMM} which also contains a
description of the I/O mechanism and MacQueen's proposal for program modules
(see later for discussion of this). Since then only few changes to the core
language have occurred.  Milner proposed equality types, and these were added,
together with a few minor adjustments~\cite{Mil3}.  The
last development before the 1990 Definition was in the exception mechanism, by MacQueen using an idea
from Burstall~\cite{AMMT}; it harmonized the ideas of exception and data type
construction.

\subsection*{Modules}

Besides contributory ideas to the core language, HOPE~\cite{BMS} contained a
simple notion of program module.  The most important and original feature of
ML modules, however, stems from the work on parameterised specifications in
CLEAR~\cite{BG}.  MacQueen, who was a member of Burstall's group at the time,
designed~\cite{Mac} a new parametric module feature for HOPE inspired by the
CLEAR work.  He later extended the parameterisation ideas by a novel method of
specifying sharing of components among the structure parameters of a functor,
and produced a draft design which accommodated features already present in ML
-- in particular the polymorphic type system.  This design was discussed in
detail at Edinburgh, leading to MacQueen's first report on modules~\cite{HMM}.

Thereafter, the design came under close scrutiny through a draft operational
static semantics and prototype implementation of it by Harper, through Kevin
Mitchell's implementation of the evaluation, through a denotational semantics
written by Don Sannella, and then through further work on operational
semantics by Harper, Milner, and Tofte.  (More is said about this in the later
section on Semantics.)  In all of this work the central ideas withstood
scrutiny, while it also became clear that there were gaps in the design and
ambiguities in interpretation.  (An example of a gap was the inability to
specify sharing between a functor argument structure and its result structure;
an example of an ambiguity was the question of whether sharing exists in a
structure over and above what is specified in the signature expression which
accompanies its declaration.)

Much discussion ensued; it was possible for a wider group to comment on
modules through using Harper's prototype implementation, while Harper, Milner
and Tofte gained understanding during development of this semantics.  In
parallel, Sannella and Tarlecki explored the implications of modules for the
methodology of program development~\cite{ST}.  Tofte, in his thesis
\cite{tofte88}, proved several technical properties of modules in a skeletal
language, which generated considerable confidence in this design. A key point
in this development was the proof of the existence of principal signatures,
and, in the careful distinction between the notion of {\it enrichment} of
structures, which allows more polymorphism and more components, and {\it
realisation} which allows more sharing.

At a meeting in Edinburgh in 1987 a choice of two designs was presented,
hinging upon whether or not a functor application should coerce its actual
argument to its argument signature.  The meeting chose coercion, and
thereafter the production of Section~\ref{statmod-sec} of this report -- the
static semantics of modules -- was a matter of detailed care.  That section is
undoubtedly the most original and demanding part of this semantics, just as
the ideas of MacQueen upon which it is based are the most far-reaching
extension to the original design of ML.

Considerable experience was gained in implementing, programming with, and
teaching the language during the nearly ten years since the definition was
first published.  Based on this experience a number of design decisions were
revisited at a meeting of the authors in Cambridge at the end of 1995.  At
this meeting it was decided to make several modest, but significant, changes
to the language in order to simplify the semantics and to correct some
shortcomings that had come to light.  The most important of these changes was
the replacement of the imperative type discipline by the so-called value
restriction (discussed above), the elimination of structure sharing as a
separate concept from type sharing, and the introduction of the closely
connected mechanisms of opaque signature matching and type abbreviations in
signatures.  An important impetus for these changes to the modules language
was the work of Leroy~\cite{leroy94}, and Harper and
Lillibridge~\cite{HL} on the type-theoretic interpretation of
modules (described below).

\subsection*{Implementation}

The first implementation of ML was by Malcolm Newey, Lockwood Morris and Robin
Milner in 1974, for the DEC10.  Later Mike Gordon and Chris Wadsworth joined;
their work was mainly in specialising ML towards machine-assisted reasoning.
Around 1980 Luca Cardelli implemented a version on VAX; his work was later
extended by Alan Mycroft, Kevin Mitchell and John Scott.  This version
contained one or two new data-type features, and was based upon the {\em
Functional Abstract Machine (FAM)}, a virtual machine which has been a
considerable stimulus to later implementation.  By providing a reasonably
efficient implementation, this work enabled the language to be taught to
students; this, in turn, prompted the idea that it could become a useful
general purpose language.

In Gothenburg, an implementation was developed by Lennart Augustsson and
Thomas Johnsson in 1982, using lazy evaluation rather than call-by-value; the
result was called {\em Lazy ML} and is described in~\cite{Aug}.  This work is
part of continuing research in many places on implementation of lazy
evaluation in pure functional languages.  But for ML, which includes
exceptions and assignment, the emphasis has been mainly upon strict evaluation
(call-by-value).

In Cambridge, in the early 1980s, Larry Paulson made considerable improvements
to the Edinburgh ML compiler, as part of his wider programme of improving {\em
Edinburgh LCF} to become {\em Cambridge LCF}~\cite{Pau}.  This system has
supported larger proofs than the Edinburgh system, and with greater
convenience; in particular, the compiled ML code ran four to five times
faster.

Around the same time G\'{e}rard Huet at INRIA (Versailles) adapted ML to
Maclisp on Multics, again for use in machine-assisted proof.  There was close
collaboration between INRIA and Cambridge in this period.  ML has undergone a
separate development in the group at INRIA on the {\em CAML}
language~\cite{CCM}.  Work on {\em CAML} included the development of several
extensions to the core language, notably updatable fields in record types,
values with dynamic types, support for lazy evaluation, and handling of
embedded languages with user-defined syntax.  It did not, however, include
modules.

The first implementation of the Standard ML core language was by Mitchell,
Mycroft and Scott at Edinburgh, around 1984.  The prototype implementation of
modules, before that part of the language settled down, was done in 1985-6;
Mitchell dealt with evaluation, while Harper tackled the elaboration (or
`signature checking') which raised problems of a kind not previously
encountered.  Harper's implementation employed a form of unification that was
later adopted in the static semantics of modules.

At around the same time the {\em Poly/ML} implementation began with a
suggestion from Mike Gordon that an interesting application of Matthews' Poly
language would be to implement Standard ML.  Important experience was gained
through Matthews' early implementation of the core language, followed by
several versions of the modules language as they were devised.  {\em Poly/ML}
features arbitrary precision arithmetic, a process package, and a windowing
system.  Considerable experience has been gained with the compiler, notably by
Larry Paulson at Cambridge and by Abstract Hardware Limited (AHL).

  The {\em Standard ML of New Jersey (SML/NJ)} system has been in active development
  since 1986~\cite{am87,am91}.
  Initially started by David MacQueen at Bell Laboratories and Andrew Appel at Princeton University,
  the project has also benefited from significant contributions
  by Matthias Blume, Emden Gansner, Lal George, John Reppy and Zhong Shao.
  {\em SML/NJ} is a robust and complete environment for Standard ML that supports
  the implementation of large software systems and generates efficient code for a
  number of different hardware and software platforms.  {\em SML/NJ} also serves as
  a laboratory for compiler research: in implementations of module systems for ML;
  code optimization based on continuation-passing style; efficient pattern matching;
  and very fast heap allocation and garbage collection.  Dozens of researchers
  have contributed to the development of the compiler, in such areas as
  efficient closure representations, first-class continuations, type-directed
  compilation, concurrent programming, portable code generators, separate
  compilation, and register allocation.
  {\em SML/NJ} has also been widely used to explore extending SML with
  concurrency features.

%In 1986 Andrew Appel and David MacQueen began work on the {\em Standard ML of
%New Jersey (SML/NJ)} compiler~\cite{am87}.  {\em SML/NJ} is a robust and
%complete environment for Standard ML that supports the implementation of large
%software systems and generates efficient code for a number of different
%hardware and software platforms.  {\em SML/NJ} also serves as a laboratory for
%compiler research: in implementations of module systems for ML; code
%optimization based on continuation-passing style; efficient pattern matching;
%and very fast heap allocation and garbage collection.  Dozens of researchers
%have contributed to the development of the compiler, in such areas as
%efficient closure representations, first-class continuations, type-directed
%compilation, concurrent programming, portable code generators, separate
%compilation, and register allocation.


In 1989, Mads Tofte, Nick Rothwell and David N. Turner started work on the
{\em ML Kit Compiler} in Edinburgh. The {\em ML Kit} is a direct translation
of the 1990 Definition into a collection of Standard ML modules, emphasis being
on clarity rather than efficiency. During 1992 and 1993, Version~1 of the {\em
ML Kit} was completed, mostly through the work of Nick Rothwell at Edinburgh
and Lars Birkedal at DIKU\cite{BRTT}. In 1994, region inference was added to
the {\em ML Kit}, by Mads Tofte. Lars Birkedal wrote a region-based C-code
generator and a runtime system in C.  In 1995, Martin Elsman and Niels
Hallenberg extended this work to generate native code for the HP PA-RISC
architecture.

Harlequin Ltd. began the implementation of a commercial compiler in 1990.  The
{\em MLWorks} system is a fully-featured graphical programming environment,
including an interactive debugger, inspector, browser, extensive profiling
facilities, separate compilation and delivery, a foreign-language interface,
and libraries for threads and windowing systems.
%Harlequin was a major
%partner in the effort to standardize the revised Standard ML basis library.

{\em Caml Light}, a lightweight reimplementation of {\em CAML} released in
1991, added a simple module system in the style of Modula-2, targeted towards
separate compilation of modules: structures and signatures are identified with
files, functors and multiple views of a structure are not supported.  These
were added in the {\em Caml Special Light} implementation in 1995, while
preserving the support for separate compilation. {\em Caml Special Light} and
the present version of Standard ML share several important simplifications,
such as the value restriction on polymorphism, type definitions in signatures,
and the lack of support for structure sharing.  The static semantics for {\em
Caml Special Light} is based on the type-theoretic properties of dependent
function types (functor signatures) and manifest types (type definitions in
signatures)~\cite{leroy94}.

{\em Moscow ML} is an implementation of core Standard ML, created in 1994 by
Sergei Romanenko in Moscow and Peter Sestoft in Copenhagen.  The {\em Caml
Light} system was used to implement the dynamic semantics, and the ML Kit
guided the implementation of the static semantics.  The result is a compact
and robust implementation, suitable for teaching.

The {\em TIL (Typed Intermediate Languages)} compiler developed at Carnegie
Mellon
University by Greg Morrisett, David Tarditi, Perry Cheng, Chris Stone,
Robert Harper,
and Peter Lee demonstrates the use of
types in compilation.  All but the last few stages of {\em TIL} are expressed
as type-directed and type-preserving transforms.  Types are used at run time
to support unboxed, untagged data representations and natural calling
conventions in the presence of variable types and garbage collection.  {\em
TIL} employs a wide variety of conventional functional language optimizations
found in other SML compilers, as well as a set of loop-oriented optimizations.
A description of the compiler and an analysis of its performance appears
in~\cite{Tar}.

Other currently active implementations are by Michael Hedlund at the
Rutherford-Appleton Laboratory, by Robert Duncan, Simon Nichols and Aaron
Sloman at the University of Sussex ({\em POPLOG}) and by Malcolm Newey and his
group at the Australian National University.

\subsection*{Semantics}

The description of the first version of ML~\cite{GMW} was informal, and in an
operational style; around the same time a denotational semantics was written,
but never published, by Mike Gordon and Robin Milner.  Meanwhile structured
operational semantics, presented as an inference system, was gaining credence
as a tractable medium.  This originates with the reduction rules of
$\lambda$-calculus, but was developed more widely through the work of Plotkin
\cite{Plo}, and also by Milner.  This was at first only used for dynamic
semantics, but later the benefit of using inference systems for both static
and dynamic semantics became apparent.  This advantage was realised when
Gilles Kahn and his group at INRIA were able to execute early versions of both
forms of semantics for the ML core language using their Typol system
\cite{Des}.  The static and dynamic semantics of the core language reached a
final form mostly through work by Tofte and Milner.

The modules of ML presented little difficulty as far as dynamic semantics is
concerned, but the static semantics of modules was a concerted effort by
several people.  MacQueen's original informal description~\cite{HMM} was the
starting point; Sannella wrote a denotational semantics for several versions,
which showed that several issues had not been settled by the informal
description.  Robert Harper, while writing the first implementation of
modules, made the first draft of the static semantics.  Harper's version made
clear the importance of structure names; work by Milner and Tofte introduced
further ideas including realisation; thereafter a concerted effort by all
three led to several suggestions for modification of the language, and a small
range of alternative interpretations; these were assessed in discussion with
MacQueen, and more widely with the principal users of the language, and an
agreed form was reached.

Concurrently with the formulation of the Definition of Standard ML, Harper and
Mitchell took up the challenge adumbrated by MacQueen~\cite{Mac2} to find
a type-theoretic interpretation of Standard ML~\cite{HM}.  This work led to
the formulation of the XML language, an explicitly-typed $\lambda$-calculus
that captured many aspects of Standard ML.  Although incomplete, their
approach formed the basis for a number of subsequent studies, including the
work of Harper and Lillibridge~\cite{HL} and Leroy~\cite{leroy94} on the
type-theoretic interpretation of modules.  This work influenced the decision
to revise the language, and culminated in a type-theoretic interpretation of
the present language by Harper and Stone~\cite{SH}.  The TIL/ML compiler
(described above) is based directly on this interpretation.

There is no doubt that the interaction between design and semantic description
of modules has been one of the most striking phases in the entire language
development, leading (in the opinion of those involved) to a high degree of
confidence both in the language and in the semantics.


\subsection*{Program Libraries}
%During 1989-1991, Dave Berry produced the first program library for
%Standard ML\cite{mllib91,berry93}. Subsequently, a partnership between
%the originators of {\em SML/NJ}, {\em MLWorks} and
%{\em Moscow ML} was formed, with the goal of creating an industrial strength
%initial basis for Standard ML. The resulting SML Basis
%Library\cite{sml-basis-lib} is a much improved and extended
%replacement of the initial basis defined in the 1990 Definition of
%Standard ML.
 During 1989-1991, Dave Berry produced the first program library for
 Standard ML\cite{mllib91,berry93}.
 The {\em SML/NJ} system is distributed with a rich library organised by
 Emden Gansner and John Reppy; this library was the starting point for
 the SML Basis Library~.
 The {\em SML Basis Library\/}\cite{sml-basis-lib} has been developed
 over the past three years in a
 partnership between the SML/NJ effort, {\em MLWorks}, and {\em Moscow ML}.
 The resulting library is a much improved and extended
 replacement of the initial basis defined in the 1990 Definition of
 Standard ML.

\subsection*{\protect\color{\addcolor} Successor ML}
{\color{\addcolor}
In 2005, an effort began to ``evolve'' Standard ML; Bob Harper suggested that the resulting language
be called Successor ML.
}

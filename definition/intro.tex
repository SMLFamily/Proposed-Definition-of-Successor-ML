%!TEX root = root.tex
%
\section{Introduction}

This document formally defines \REPL{Successor }{Standard} ML.
\ADD{It is derived from the 1997 \emph{Definition of Standard ML} by adding the
changes suggested by Andreas Rossberg in the HaMLet S documentation.}

To understand the method of definition, at least in broad terms, it helps to
consider how an implementation of ML is naturally
organised.  ML is an interactive
language,\index{4.1}
and a {\sl program}\index{4.2}  consists of a sequence of {\sl top-level
declarations};\index{4.3} the execution
of each declaration modifies the top-level environment, which we call
a {\sl basis}, and reports the modification to the user.

In the execution of a declaration there are three phases:
{\sl parsing}, {\sl elaboration}, and {\sl evaluation}.
Parsing
determines the grammatical form of a declaration.  Elaboration, the
{\sl static} phase, determines whether it is well-typed and
well-formed in other ways, and records relevant type or form information
in the basis. Finally evaluation, the {\sl dynamic} phase, determines the
value of the declaration and records relevant value information in the
basis.  Corresponding to these phases, our formal definition divides
into three parts:  grammatical rules, elaboration rules, and evaluation
rules.  Furthermore, the basis is divided into the {\sl static}
basis and the {\sl dynamic} basis; for example, a variable which has been
declared is associated with a type in the static basis and with a value in
the dynamic basis.

In an implementation, the basis need not be so divided.  But for the
purpose of formal definition, it eases presentation and understanding to
keep the static and dynamic parts of the basis separate.
This is further justified by programming experience.  A large proportion
of errors in ML programs are discovered during elaboration, and identified
as errors of type or form, so it follows that it is useful to perform
the elaboration phase separately.  In fact, elaboration without
evaluation is \replacement{\thenewpreface}{just}{part of} what is normally called {\sl compilation};
once
a declaration (or larger entity) is compiled one wishes to evaluate it --
repeatedly -- without re-elaboration, from which it follows that it is
useful to perform the evaluation phase separately.

A further factoring of the formal definition is possible,
because of the structure of the language.  ML consists of a lower level
called the {\sl Core language} (or {\sl Core} for short), a middle level
concerned  with programming-in-the-large called {\sl Modules},
and a very small upper level called {\sl Programs}.
With the three phases described above, there is therefore
a possibility of nine components in the
complete language definition. We have allotted one section to each
of these components, except that we have combined the parsing,
elaboration and evaluation of Programs in one section. The
scheme for the ensuing seven sections is therefore as follows:

\vspace*{3mm}
\begin{tabular}{rccc}
                & {\em Core} & {\em Modules} & {\em Programs} \\
\cline{2-4}
{\em Syntax}    &\multicolumn{1}{|c}{Section 2}
                             &\multicolumn{1}{|c}{Section 3}
                                             &\multicolumn{1}{|c|}{ }\\
\cline{2-3}
{\em Static Semantics}
                &\multicolumn{1}{|c}{Section 4}
                             &\multicolumn{1}{|c}{Section 5}
                                             &\multicolumn{1}{|c|}{Section 8}\\
\cline{2-3}
{\em Dynamic Semantics}
                &\multicolumn{1}{|c}{Section 6}
                             &\multicolumn{1}{|c}{Section 7}
                                             &\multicolumn{1}{|c|}{}\\
\cline{2-4}
\end{tabular}
\vspace*{3mm}

%The Core is
%a complete language in its own right, and its embedding in the full
%language is simple;  therefore each of the three parts of the formal
%description is further divided into two -- one for the Core, and one for
%Modules.

The Core provides many phrase classes, for programming convenience.
But about half of these classes are derived forms, whose meaning can be
given by translation into the other half which we call the
{\sl Bare} language.
Thus each of the three parts for the Core treats only the bare language;
the derived forms are treated in  Appendix~\ref{derived-forms-app}.
This appendix also contains a few derived forms for Modules.
A full grammar for the language is presented in
Appendix~\ref{core-gram-app}.

In\index{5.0} Appendices~\ref{init-stat-bas-app} and~\ref{init-dyn-bas-app}
the {\sl initial basis} is detailed.  This basis,
divided into its static and dynamic parts, contains the static and
dynamic meanings of \replacement{\thelibrary}{all predefined identifiers.}{a
small set of predefined identifiers. A richer basis is defined
in a separate document\cite{sml-basis-lib}.}

The semantics is presented in a form  known as Natural
Semantics.\index{5.1}  It consists of a set of rules allowing
{\sl sentences} of the form
\[ A \vdash phrase \Rightarrow A' \]
to be inferred, where $A$ is often a basis (static or dynamic) and $A'$ a
semantic object
-- often a type in the static semantics and a value in the dynamic
semantics. One should read such a sentence as follows: ``against
the background provided by
$A$, the phrase $phrase$ elaborates -- or evaluates -- to the object
$A'$''.
Although the rules themselves are formal the semantic
objects, particularly the static ones, are the subject of a mathematical
theory which is presented in a succinct form in the relevant sections.\deletion{\thenostrsharing}{
This theory, particularly the theory of types and signatures, will
benefit from a more pedagogic treatment in other publications; the
treatment here is
probably the minimum required to understand the meaning of the rules.}

The robustness of the semantics depends upon theorems.
Usually these have been proven, but the proof is not included.
\deletion{\thenostrsharing}{In two cases, however, they are presented as ``claims'' rather than
theorems; these are the claim of principal environments in
Section~\ref{principal-env-sec}, and the claim of principal signatures
in Section~\ref{prinsig-sec}. We need further confirmation of our
detailed proofs of these claims, before asserting them as theorems.}
